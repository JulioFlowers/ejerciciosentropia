
Se puede apreciar que el gas al ser expandido isotérmicamente sufrió un proceso reversible, esto debido a que
conocemos el camino por el cual llego a esa diferencia de volúmenes.

\begin{enumerate}[label=\textbf{\alph*)}]
    \item{De la primer ley de la termodinámica, tenemos para una isoterma que:
    
    \begin{gather*}
        \Delta U\ =\ 0\ \ \ \Rightarrow\\
        Q\ =\ -W\ \ \ \therefore\\
        \\
        Q\ =\ \int_{V_{i}}^{V_{f}} p\ dV\ =\ nRT_{i} \int_{V_{i}}^{V_{f}} \frac{1}{V}\ dV\ =\ 
        \boxed{nRT_{i} \ln{\left(\frac{V_{f}}{V_{i}}\right)}}
    \end{gather*}
     } 

    \item{ al ser un proceso reversible tenemos la siguiente relación 
    
    \begin{equation}
        \int \frac{dQ}{T}\ =\ \int dS\ \ \ \Rightarrow\ \ \  \Delta S\ =\ \frac{Q}{T}
    \end{equation}
    
    Como tenemos un proceso isotérmico 

    \begin{equation*}   
        \boxed{\Delta S\ =\ nR\ln{\left(\frac{V_{f}}{V_{i}}\right)}}
    \end{equation*}
    }
\end{enumerate}