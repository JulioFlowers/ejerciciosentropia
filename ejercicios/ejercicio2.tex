
Podemos apreciar que este sistema se encuentra a una presión constante
pues se ebulle a presión atmosférica por lo tanto podemos aprovechar 
la definición de C\textsubscript{p}.

\begin{gather*}
    C_{p}\ =\ \frac{\dbar Q_{p}}{mdT}\ \ \ \Rightarrow\ \ \
    \dbar Q_{p}\ =\ mC_{p}dT\ \ \ \Rightarrow\\
    dS_{p}\ =\ \frac{mC_{p}}{T}\ dT\ \ \ \therefore\\
    \\
    \Delta S\ = mC_{p} \int_{273.15\ \um{K}}^{373.15\ \um{K}}\ \frac{1}{T}\ dT\ =\
    mC_{p} \ln{\left(\frac{373.15\ \um{K}}{273.15\ \um{K}}\right)}\\ 
    \\
    \shortintertext{utilizando el valor de C\textsubscript{p} registrado por Engineering ToolBox\cita{Engineering}}\\
    \approx \ 1\ kg\ 4.216 \left[\frac{kJ}{K\ kg}\right] 0.312\ \approx\ \boxed{1.315\ \um{kJ\ K^{-1}}}
\end{gather*}