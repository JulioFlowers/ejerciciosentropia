\begin{enumerate}
    \item  \includegraphics*[H]{T-S.png}
    \item  Como se trata de un proceso reversible  sabemos que:
    \[ \dbar Q =TdS \]
   El proceso de $A$ a $B$ se trata de un proceso isotermico, por lo que si integramos la expresión anterior tenemos:
   \[ \int \dbar Q =T_{c}\int dS \]
   \[ \Rightarrow \int _{v_A} ^{v_B} \dbar Q = T_{c} \Delta S\]
   \[nRT_c \ln \left( \frac{v_B}{v_A}\right) = T_{c}\Delta S = Q_{entrada}\]
    En el proceso de $B$ a $C$ al tratarse de un proceso adiabático, no hay intercambio de calor $\Rightarrow Q=0$\\

    En  el proceso de $C$ a $D$ tenemos un proceso isotermico, por lo que analogamente al proceso de $A$ a $B$ tenemos:

\[ \int \dbar Q =T_{f}\int dS \]
   \[ \Rightarrow \int _{v_C} ^{v_D} \dbar Q = T_{f} \Delta S\]
   \[nRT_f \ln \left( \frac{v_D}{v_C}\right) = T_{f}\Delta S = Q_{salida}\]

Como se trata de un calor de salida 
\[Q_{salida}=-nRT_c \ln \left( \frac{v_D}{v_C}\right)\]
  

   Ahora bien si  se trata de un gas ideal y el proceso se realiza a presión constante tenemos que:
   \[ \frac{v_1}{T_1}=\frac{v_2}{T_2}\]
   \[\Rightarrow \frac{T_2}{T_1}=\frac{v_2}{v_1}\]

Entonces tenemos que:
\[ \frac{v_B}{v_A}=\frac{Tf}{Tc}=\frac{v_D}{v_c}\]

Tenemos que la eficiencia esta dada por:
\[ \eta = \frac{Q_{entrada}+ Q_{salida}}{Q_{entrada}}\]

\[\Rightarrow  \eta = \frac{nRT_c \ln \left( \frac{v_B}{v_A}\right)-nRT_c \ln \left( \frac{v_D}{v_C}\right)}{nRT_c \ln \left( \frac{v_B}{v_A}\right)}\]
\[\Rightarrow  \eta = \frac{nRT_c \ln \left( \frac{v_B}{v_A}\right)-nRT_c \ln \left( \frac{v_B}{v_A}\right)}{nRT_c \ln \left( \frac{v_B}{v_A}\right)}\]
\[\eta =\frac{T_c-T_f}{T_c}\]





\end{enumerate}
